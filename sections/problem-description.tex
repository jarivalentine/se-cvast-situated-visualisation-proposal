\section{Problem Description (Why?)}

\subsection{Introduction}

Situated visualization is a situated data representation that is purely visual and typically on a screen. Here, situated data representation is data presentation where the presentation itself must be somewhat close to the data's physical referent \cite{WYP17}.
The concept is relevant in multiple fields including HCI and information visualization \cite{WYP17} and is also extensively being explored in areas such as augmented reality and ubiquitous computing \cite{BKT+22}.

One approach to situated visualization can be found in, for example, a physical site visit. An urban planner might use a system like SiteLens to visualize data directly within the real world.
For instance, overlaying carbon monoxide (CO) readings, which are collected in real-time, onto the street to reveal the higher CO levels correspond to cars idling there \cite{WF09}.
A comprehensive overview of available approaches lack within this field.

\subsection{Context}

Beyond the relevant areas mentioned above, we can broaden where this topic belongs to outside of traditional research fields.
Situated visualization is most definitely part of the trend toward virtual and augmented reality.
VR technology has been growing steadily, becoming better and better, and expanding beyond entertainment as the desire to use it for more purposeful applications becomes strong.

\subsection{Status Quo}

\cite{BKT+22} worked on an extensive list of papers to conclude perspectives on situated visualization and found that the most broad definition is by White and Feiner \cite{WF09}. They state situated visualization ``is related to and displayed in its environment".
They also found Willet et al. to build on that work introducing concepts like the ``physical data referent" as mentioned previously \cite{WYP17}.
While the first definition dates back from 2009 the latter is more recent, from 2022, and has since been commonly used to define situated visualization in research.

Willet et al. also makes a distinction between situated and embedded visualizations \cite{WYP17} which makes us ask the following questions.
What definition should we follow?
Are the same definitions used outside of research, where we will find exactly the type of approaches which might be insightful? And what other terms might be used to refer to basically the same concept?
In practice we find, though not limited to, the following terms: on-site, in-situ, ambient, ubiquitous, location-based, embodied, embedded, contextual.

\subsection{What is the problem}

There is currently no structured overview of all approaches.
Furthermore, Bressa et al. highlight the second issue: ``This wide appropriation of situated visualization as a research concept has led to a disconnected terminology..." and ``As a result, interpretations of what situatedness and situated visualization are, and how these concepts are understood in the current literature, remain unclear.'' \cite{BKT+22}.
Similarly Willett et al. states that the term `situated' has been previously used in different ways, before the widespread adoption of their definition. And that ``the benefits, trade-offs, and even the language necessary to describe and compare different approaches remains limited.''. \cite{WYP17}

Beyond the term ``situated visualization'', we could argue that the concept dates back to the 1970, when te U.S. Department of Defense starting operating their GPS system (alongside the USSR's GLONASS). This was emphasized by Spiekermann back in 2004, when the term ``location-based services'' was used to describe services that integrate a mobile device's location with other data to provide value to a user. \cite{SV04}
While today the term `location-based' might directly by used to refer to situated visualization approaches.

\subsection{Why is it a problem?}

Without a clear overview of all approaches, designers have to independently looking into what exists, encountering the same problems that could be avoided by documenting the findings systematically.

When terminology shifts, we argue that previous explorations of similar concepts easily get neglected, once again forcing researchers to repeat challenges that might have already been documented long ago.

\subsection{What is unknown?}

Beyond what has been stated before, we now look into specific knowledge gaps that could be included as metadata within a comprehensive overview. We can ask questions such as:
\begin{enumerate}
    \item Does the approach raise any ethical concerns?
    \item Can we categories or cluster certain approaches?
    \item What is its predecessor/which approaches is this a predecessor to?
\end{enumerate}
These are all self-explanatory in how it can help a user of the overview decide on what approach suits their application best.

\subsection{What could be improved?}

We can not systematically change the way we use definitions and terms, nor can we change how concepts were defined or referred to in the past. But what we \textit{can do} is account for all the concerns by e.g. considering every term we find and documenting their use in a way we can understand how meanings shifted.

As Bawden (2001) states, ``...any terminology built around this central concept is in danger of being constructed on `shifting sands''' \cite{Baw01}. To at least try and come close to an exhaustive overview, it is therefore important to consider previously used terminologies, as these reveal insights and boundaries. Although this concern can be raised for almost any research topic when creating an overview, it is especially important within situated visualization, since inconsistent terminology frequently arises, which is also highlight by several major papers in this field .

\subsection{What is the goal of the project?}

The main goal of this project is to provide a structured and comprehensive state-of-the-art overview of available approaches to situated visualization.
Furthermore, we want to eliminate the previously identified concerns by using methods that ensure clarity and consistency.
It is important for the overview to be both extensive and accessible/usable.
The following section outlines how this will be achieved.
