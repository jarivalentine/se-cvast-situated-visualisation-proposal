\section{Problem Description (Why?)}

\subsection{Introduction}

According to Willett et al. (2017), a situated data representation presents data somewhat close to the its physical data referent in the real world.
Situated visualization is a visual form of such representations that is typically shown on a screen \cite{WYP17}.
The concept is relevant in multiple fields including HCI and information visualization \cite{WYP17} and is also being extensively explored in areas such as augmented reality and ubiquitous computing \cite{BKT+22}.

One approach to situated visualization can be found in, for example, a physical site visit. An urban planner might use a system like SiteLens to visualize data directly within the real world.
For instance, overlaying carbon monoxide (CO) readings, which are collected in real-time, onto the street to reveal the higher CO levels correspond to cars idling there \cite{WF09}.

\subsection{Context}

Beyond the relevant areas mentioned above, we can broaden where this topic belongs to beyond traditional research fields.
Situated visualization is most definitely part of the ongoing trend toward augmented and mixed reality.
These technologies have been growing steadily, improving in quality, and expanding beyond entertainment as the desire to use it for more purposeful applications becomes strong.

\subsection{Status Quo}

Bressa et al. (2022) analyzed an extensive list of papers to conclude perspectives on situated visualization \cite{BKT+22} and found that the most broad definition is by White and Feiner (2009). They state situated visualization ``is related to and displayed in its environment'' \cite{WF09}.
They also found Willett et al. to build on that work introducing concepts like the ``physical data referent'' as mentioned previously \cite{WYP17}.
While the first definition dates back from 2009, the latter is more recent, from 2017, and has since been commonly used to define situated visualization in research.

Willett et al. also make a distinction between situated and embedded visualizations \cite{WYP17} which makes us ask the following questions:
What definition should we follow?
And what other terms might be used to refer to basically the same concept?
In practice we find, though not limited to, the following terms: on-site, in-situ, ambient, ubiquitous, location-based, embodied, embedded, contextual \cite{POB+25}.

\subsection{What is the problem}

Research literature highlight that the terminology surrounding situated visualization is inconsistently used.
Bressa et al. note that: ``This wide appropriation of situated visualization as a research concept has led to a disconnected terminology..." and ``As a result, interpretations of what situatedness and situated visualization are, and how these concepts are understood in the current literature, remain unclear.'' \cite{BKT+22}.
Similarly Willett et al. states that the term `situated' has been previously used in different ways, before the widespread adoption of their definition. And that ``the benefits, trade-offs, and even the language necessary to describe and compare different approaches remains limited.'' \cite{WYP17}.

This inconsistency is further complicated by historical evolution of related concepts. We could argue that the concept dates back to the 1970s, when te U.S. Department of Defense started operating their GPS (alongside the USSR's GLONASS). This was emphasized by Spiekermann back in 2004, when the term ``location-based services'' was used to describe services that integrate a mobile device's location with other data to provide value to a user \cite{SV04}.
While today the term `location-based' might directly be used to refer to situated visualization approaches.

While there are overviews on situated visualization today, they do not cover the full range of approaches that have emerged in the years up to 2025.

\subsection{Why is it a problem?}

Without a clear overview of all approaches, designers have to independently look into what exists, encountering the same problems that could be avoided by documenting the findings systematically.

When terminology shifts, we argue that previous explorations of similar concepts easily get neglected, once again forcing researchers to repeat challenges and discover insights that might have already been documented long ago.

\subsection{What is unknown?}

Beyond what has been stated before, several knowledge gaps remain that could be explored within a comprehensive overview.
These include: categorization of approaches, relationships between predecessors and successors, and importantly, ethical considerations.

For the scope of this seminar paper, the focus will be on the ethical aspects of situated visualization approaches.
This aligns with the the framework of Responsible Research and Innovation (RRI), where ethical reflection is not considered optional but a requirement.
As von Schomberg (2013) defines RRI, innovation must strive for “ethical acceptability, sustainability and societal desirability” \cite{Sch13}.

\subsection{What could be improved?}

We cannot systematically change the way we use definitions and terms, nor can we change how concepts were defined or referred to in the past. But what we \textit{can do} is account for all the concerns by e.g. considering every term we find and documenting their use in a way we can understand how meanings shifted.

As Bawden (2001) states, ``...any terminology built around this central concept is in danger of being constructed on `shifting sands'.'' \cite{Baw01}.
To at least try and come close to an exhaustive overview, it is therefore important to consider previously used terminologies, as these reveal insights and boundaries.
Although this concern can be raised for almost any research topic when creating an overview, it is especially important within situated visualization, since inconsistent terminology frequently arises, which is also highlighted by several major papers in this field such as \cite{WYP17} \cite{BKT+22} (as mentioned in section 1.4).

\subsection{What is the goal of the project?}

The main goal of this project is to initiate what could become an exhaustive overview of situated visualization.
Since creating a comprehensive overview is beyond the limits of this seminar, the project will focus on a recent five-year period as a starting point.
Within this scope, the project aims to address the previously identified concerns by using methods that ensure clarity and consistency, with a specific focus on documenting the ethical considerations of each approach. The following section outlines how this will be achieved.
