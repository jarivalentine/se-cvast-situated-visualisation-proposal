\section{Problem Description (Why?)}

% - Introduction
% - Context
% - Status Quo
% - What is the problem?
% - Why is it a problem?
% - What is unknown?
% - What could be improved?
% - What is the goal of the project?

\subsection{Introduction}

% What is situated visualisation?
Situated visualization is defined by \cite{} as "a situated data representation for which the presentation is purely visual — and is typically on a screen."

Why is it relevent within HCI and data visualisation
Exmaples (manufacturing, maintance, AI-assisted site visits)
A comprehensive overview lacks withint this field.

Context

Where does the topic belong (broaden):
part of the trend toward VR and AR and context-aware data representation within HCI and visualisation.
Growing VR tech, becoming better and better, the need and will to use those thech for purpose and not just entertainment.
The gravitation towards AR, not wanting to leave our own world, having it add on to our own world.

Status Quo

WHAT EXISTS
What is currently known.
What pappers discuess situated visualization?
Different terms: embedded, physical visualisation
There is no unified overview that compares or catogerazies exciting approaches.

What is the problem

WHAT'S MISSING
Despite existing studies, and a lot of work withint this field,
there is no map or structured overview of all the areas and methods used.
Communicate with advisor

Why is it a problem?

WHY THAT MATTERS (THAT IT's MISSING)
Not just that something is missing, but why it matters that this is missing.
Communicate with advisor

What is unknown?

SPECIFY KNOWLEDGE GAP
Beyond what has already been said, no overview of situatued visuation techniques.
What else which is unclear is further how these approaches differ in purpose, interaction or context.
How can we catagorize? How can we map them out and see where they connect or differ?
What is being developped/discovered currently?
Are their ethical and privacy problems with some of the innovations?

What could be improved?

WHAT THIS WILL CONTRIBUTE
Creating an overview would clarify terminology, catogories and help research opportunities.
Can we rate approaches to situated visualisation?
Can we label what problems each approaches is best a solving?
Add a timeline of approaches and what caused it to come about in that time.
We then can understand why certain approaches got here which helps us go back to its purpose,
then again a technologies initial purpose can completely be forgotten by the current usage,
we can compare and see how we might learn from the past.
Can we rank them based on their risks or how much data they'd need/consider privacy (would GDPR be happy?) and ethics.
A unified overview, which in theory comes close to the effects of a standard can help us all communicate in the same language.

What is the goal of the project?

WHAT THIS WILL CONTRIBUTE
The goal of this work is to provide a structed state-of-the-art overview of approahces situated visualisation
including contexts. methods, domains, etc.

What approaches empose dangers, can we try and be responsible and predict the dangers of our technologies?
In contrast, situated visualisation appraoches have been proven to be great benefit, not only in privat sectors but also in public spaces
<Cite>, has proven how this works with example...
Can we list possitive and negative examples of real world scenarios, where things have gone wrong in the past with a approach,
or in contrast how we have benefited (mainly without) any sideeffects? If any, what sideeffects where there?
With a solid overview of not only what approaches excist but also listing how we define possitive and negative examples,
the overview can help any decision being made on what approach to use.
