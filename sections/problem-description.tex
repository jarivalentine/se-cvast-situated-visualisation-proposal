\section{Problem Description (Why?)}

% - Introduction
% - Context
% - Status Quo
% - What is the problem?
% - Why is it a problem?
% - What is unknown?
% - What could be improved?
% - What is the goal of the project?

\subsection{Introduction}

% What is situated visualisation?
Situated visualization can be defined as a situated data representation that is purely visual and typically on a screen.
And situated data representation is data presentation where the presentation itself must be somewhat close to the data's physical referent \cite{WYP17}.
Keep in mind that this is only one definition of situated visualization and inconsistency could be a potential problem we need to discuss.

% why is it relevant to: our Institute of Visual Computing and Human-Centered Technology, the Centre for Visual Analytics Science and Technology, specifically Information Visualzation and Visual Analytics, and my programme Media and Human-Centered Computing

"The concept has gained interest from multiple research communities, including visualization,
human-computer interaction (HCI) and augmented reality." \cite{WYP17}

“Researchers in related areas such as ubiquitous computing (UbiComp) [48] and ambient/ peripheral displays [53, 40] have created a variety of physical information displays.” \cite{WYP17}

“Examples of situated and embedded data representations have been independently explored in a variety of research areas, including information visualization, augmented reality, and ubiquitous computing.” \cite{BKT+22}

“The area of situated visualization follows a strand of research agendas in visualization that are concerned with moving beyond traditional desktop applications [75] such as ‘Ubiquitous Analytics’ [26], ‘Immersive Analytics’ [3, 29], or ‘Situated Analytics’ [27, 90].” \cite{BKT+22}

Exmaples (manufacturing, maintance, AI-assisted site visits)

A comprehensive overview lacks withint this field.

\subsection{Context}

% Where does the topic belong (broaden):
as mentioned above
and part of the trend toward VR and AR and context-aware data representation within HCI and visualisation.
Growing VR tech, becoming better and better, the need and will to use those thech for purpose and not just entertainment.
The gravitation towards AR, not wanting to leave our own world, having it add on to our own world.

\subsection{Status Quo}

% WHAT EXISTS
% What is currently known.

\cite{BKT+22} worked on an exhaustive list of papers to conclude perspectives on situated visualisation and found that the most broad definition is by White and Feiner \cite{WF09}: situated visualization "is related to and displayed in its environment".
They also found Willet et al. build on that work introducing concepts as "physical data referent" as mentioned previously \cite{WYP17}.
While the first definition dates back from 2009 the latter is more recent, from 2022, and has been commonly used to define SV in research.
Willet et al. also make a distinction between situated and embedded visualisations which again introduces the problem of what definition should we use? Where does situated visualization stop?
In practice we find, though not limited to, the following terms: on-site, in-situ, ambient, ubiquitous, location-based, embodied, embedded, prosemic, contextual.

\subsection{What is the problem}

% WHAT'S MISSING
there is no structured overview of all approaches.
there are many terms that in pratice mean the same thing.

\subsection{Why is it a problem?}

% WHY THAT MATTERS (THAT IT's MISSING)
% Not just that something is missing, but why it matters that this is missing.

We could use an overview that lets use consider every options withint a glance.
Without needing to dive into research each time.
We want to known where terms connect so we can compare approaches eventhough they were initially defined diffently.

\subsection{What is unknown?}

% SPECIFY KNOWLEDGE GAP
Beyond what has already been said, no overview of situatued visuation techniques.
What else which is unclear is further how these approaches differ in purpose, interaction or context.
How can we catagorize? How can we map them out and see where they connect or differ?
What is being developped/discovered currently?
Are their ethical and privacy problems with some of the innovations?

\subsection{What could be improved?}

% WHAT THIS WILL CONTRIBUTE
Creating an overview would clarify terminology, catogories and help research opportunities.
Can we rate approaches to situated visualisation?
Can we label what problems each approaches is best a solving?
Add a timeline of approaches and what caused it to come about in that time.
We then can understand why certain approaches got here which helps us go back to its purpose,
then again a technologies initial purpose can completely be forgotten by the current usage,
we can compare and see how we might learn from the past.
Can we rank them based on their risks or how much data they'd need/consider privacy (would GDPR be happy?) and ethics.
A unified overview, which in theory comes close to the effects of a standard can help us all communicate in the same language.

\subsection{What is the goal of the project?}

% WHAT THIS WILL CONTRIBUTE
The goal of this work is to provide a structed state-of-the-art overview of approahces situated visualisation
including positive/negative examples, ethical considerations, initial prupose - current purpose, visualized in a timeline.

What approaches empose dangers, can we try and be responsible and predict the dangers of our technologies?
In contrast, situated visualisation appraoches have been proven to be great benefit, not only in privat sectors but also in public spaces
<Cite>, has proven how this works with example...
Can we list possitive and negative examples of real world scenarios, where things have gone wrong in the past with a approach,
or in contrast how we have benefited (mainly without) any sideeffects? If any, what sideeffects where there?
With a solid overview of not only what approaches excist but also listing how we define possitive and negative examples,
the overview can help any decision being made on what approach to use.
