\section{Expected Results}

The main expected outcome of this project is a structured interactive overview of existing approaches to situated visualization presented along a timeline.
Using a timeline helps both a clear structure and intuitive way to explore the shift in terminology.
As Plaisant and Shneiderman (1998) highlight, ``an overview of a personal history is seen in a single screen as a series of timelines. Users can filter, zoom, highlight, and get details on demand'' \cite{PSM98}, they emphasize the power of timelines to present complex information in a usable way.

To achieve the level of flexibility, a HTML webpage will be implemented. This allows approaches to become expandable, giving users an accessible overview at first hand while enabling them to explore metadata by a single click.

Afterwards, users will be able to explore the (as exhaustive as possible) overview to help them decide which approaches suit their needs, compare approaches they're considering and analyze unfamiliar terminology within the context of situated visualization.
